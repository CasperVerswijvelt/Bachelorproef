\chapter{Stand van zaken}
\label{ch:stand-van-zaken}

% Tip: Begin elk hoofdstuk met een paragraaf inleiding die beschrijft hoe
% dit hoofdstuk past binnen het geheel van de bachelorproef. Geef in het
% bijzonder aan wat de link is met het vorige en volgende hoofdstuk.

% Pas na deze inleidende paragraaf komt de eerste sectiehoofding.

Om te de verschillende 'ontgoogle' manieren met elkaar te kunnen vergelijken, moeten we eerst wat dieper ingaan op de manieren waarop dit gedaan kan worden. In dit hoofdstuk zult u meer te weten komen over hoe een 'ontgoogling' kan gebeuren en hoe dit precies in zijn werk gaat.

\section{Google Play-Services}
Aan het hart van het Android besturingssysteem liggen de 'Google Play-Services'. Deze services zijn zeer nauw met het Android besturingssysteem geïntegreerd, en bieden een bundel van API's aan. \autocite{marshall_google-play-services}. Een API of 'Application Program Interface' is kort gezegd een set van hulpmiddelen die het voor applicatie-ontwikkelaars makkelijker maakt om applicaties te ontwikkelen. \autocite{beal_api}

AOSP, ook gekend als Android Open Source Project, is de basis van het Android besturingssysteem. Zoals de naam zelf al zegt, is dit project volledig 'open-source'. Dit houdt in dat iedereen vrij is om de broncode te gebruiken en aan te passen. De Android versie die op de meeste mensen hun smartphone draait, is echter niet volledig 'open-source' maar ook voor een deeltje 'closed-source'. Dit deel omvat dan alle applicaties van Google zelf, bv. GMail, Google Maps, en de onderliggende 'Google Play-Services'. \autocite{amadeo_open-source} AOSP biedt een bundel API's  aan, maar Google zelf adviseert om de API's van de Google Play-Services te gebruiken. Deze zouden immers een stuk makkelijker te gebruiken zijn, en zorgen ervoor dat applicaties niet rechtstreeks afhankelijk zijn van Android besturingssysteem updates. \autocite{marshall_google-play-services} Deze API's, zijn dan ook 'closed-source', tegenover de API's van AOSP, die 'open-source' zijn.  

Steeds meer en meer open-source applicaties, die voorheen te vinden waren binnen AOSP, worden achtergelaten en vervangen door een closed-source versie op de Google Play Store, de applicatie van Google waarmee gebruikers applicaties kunnen downloaden en installeren. Ook de Play Store maakt gebruik van de Google Play-Services om applicaties automatisch te kunnen updaten. Op deze manier probeert Google haar eigen applicaties zoveel mogelijk macht te geven tegenover de 'open-source' applicaties binnen AOSP. \autocite{amadeo_open-source}

Doordat zeer veel applicaties dus rechtstreeks afhangen van de Google Play-Services, zullen veel applicaties ook regelrecht niet meer zoals gewenst functioneren wanneer de Play-Services niet aanwezig zijn of deze niet beschikt over de nodige toestemmingen \autocite{marshall_google-play-services}. Dit wordt ook gemeld wanneer je de machtigingen van de Google Play-Services probeert in te perken, door een melding die zegt \textquote{Als je deze machtiging weigert, kan het zijn dat basisfuncties van je apparaat niet meer werken zoals bedoeld.}.

\section{Mogelijke ontgoogle manieren}

Het doel van het ontgooglen is dan om de 'Google Play-Services' zodanig te vermijden zodat je op geen enkel vlak nog vastzit aan Google. Dit kan mogelijks op enkele manieren gebeuren, die zullen worden besproken in volgende subsecties.

\subsection{Verwijderen van Google Play-Services}
De meest logische optie om Google te vermijden is het verwijderen van de 'Google Play-Services'. Ook dit kan op verschillende manieren bereikt worden.

\subsubsection{Installeren van een Custom ROM}
Een 'Custom Android ROM' verwijst naar de versie van AOSP waarop een bepaalde smartphone draait \autocite{custom-rom}. De versie van Android die vooraf geïnstalleerd is op een smartphone wordt de 'stock ROM' genoemd. De meeste stock ROM's bevatten standaard reeds de Google Play-Services en bijhorende Google applicaties, maar dit is zeker geen vereiste voor het android besturingssysteem. In China is het gebruik van Google verboden, en bijgevolg bevatten toestellen die bedoeld zijn voor de Chinese markt geen enkele verwijzing naar Google software. Doordat de Google Play Store niet kan werken zonder de Google Play-Services, moeten chinese bedrijven een eigen 'app store' implementeren om deze te vervangen. Hetzelfde is waar voor custom ROM's. De ontwikkelaar kan zelf kiezen of hij/zij google software wilt meeleveren in zijn versie van het besturingssysteem.

Als er binnen de custom rom geen google software te vinden is, kan een Android telefoon die deze versie van het besturingssysteem draait op dit moment als 'ontgooglet' beschouwd worden. 

Je verwijst bij elke bewering die je doet, vakterm die je introduceert, enz. naar je bronnen. In \LaTeX{} kan dat met het commando \texttt{$\backslash${textcite\{\}}} of \texttt{$\backslash${autocite\{\}}}. Als argument van - commando geef je de ``sleutel'' van een ``record'' in een bibliografische databank in het Bib\TeX{}-formaat (een tekstbestand). Als je expliciet naar de auteur verwijst in de zin, gebruik je \texttt{$\backslash${}textcite\{\}}.
Soms wil je de auteur niet expliciet vernoemen, dan gebruik je \texttt{$\backslash${}autocite\{\}}. In de volgende paragraaf een voorbeeld van elk.

\textcite{Knuth1998} schreef een van de standaardwerken over sorteer- en zoekalgoritmen. Experten zijn het erover eens dat cloud computing een interessante opportuniteit vormen, zowel voor gebruikers als voor dienstverleners op vlak van informatietechnologie~\autocite{Creeger2009}.
