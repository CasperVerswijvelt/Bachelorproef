%%=============================================================================
%% Conclusie
%%=============================================================================

\chapter{Conclusie}
\label{ch:conclusie}

% TODO: Trek een duidelijke conclusie, in de vorm van een antwoord op de
% onderzoeksvra(a)g(en). Wat was jouw bijdrage aan het onderzoeksdomein en
% hoe biedt dit meerwaarde aan het vakgebied/doelgroep? 
% Reflecteer kritisch over het resultaat. In Engelse teksten wordt deze sectie
% ``Discussion'' genoemd. Had je deze uitkomst verwacht? Zijn er zaken die nog
% niet duidelijk zijn?
% Heeft het onderzoek geleid tot nieuwe vragen die uitnodigen tot verder 
%onderzoek?

\section{Stappenplannen testgevallen}
In het algemeen werd er niet veel nieuwe informatie ontdekt binnen dit onderdeel van het experiment. Het meeste van deze informatie is namelijk ook te vinden op het internet. De uitgeschreven stappen bieden echter wel genoeg informatie voor gebruikers die door dit onderzoek overtuigd zijn om de macht van Google in te perken op hun apparaat. In volgende subsecties worden de resultaten per testgeval in detail besproken.


\subsection{Testgeval 1: Android met fabrieksinstellingen}
Uit de resultaten van de uitgeschreven stappenplannen kan afgeleid worden dat het niet veel moeite vereist om de toestand van testgeval 1 te bekomen. Dit is namlijk ook de standaarinstelling van het apparaat. Alle instellingen kunnen op het apparaat zelf worden uitgevoerd en er is geen behoefte aan externe apparatuur of software.

\subsection{Testgeval 2: Android met aangepaste instellingen}
Om testgeval 2 te bereiken, moesten er bovenop het stappenplan van testgeval 1 enkele instellingen worden aangepast. Deze instellingen zelf zijn niet direct makkelijk te vinden en staan verspreid op het apparaat. Met het uitgeschreven stappenplan is het echter wel duidelijk wat er moet gebeuren. Alle instellingen kunnen op het apparaat zelf gebeuren en er is geen behoefte aan externe apparatuur of software.

\subsection{Testgeval 3: Aangepaste versie van Android}
Binnen dit geval is het uitgeschreven stappenplan al direct groter dan deze van de andere testgevallen. Er worden binnen dit testgeval ook gebruik gemaakt van extra software, beide op het apparaat zelf en op de computer die gebruikt wordt om de software op het Android apparaat op te laden. Deze software wordt niet beschikbaar gesteld door Google of Android zelf, maar door externe ontwikkelaars. Doordat de software niet op een centrale plek wordt vrijgegeven, is het goed mogelijk dat een gebruiker een foute of illegitieme versie binnenhaalt. Foute software kan mogelijk schade veroorzaken aan de computer die gebruikt wordt, of zorgen dat het Android apparaat vastloopt en niet meer te gebruiken is. Als de juiste software gebruikt wordt, moet de gebruiker nog steeds specifieke commando's invoeren in een 'terminal' of 'cmd' venster. Voor normale gebruikers is dit niet gebruiksvriendelijk. De kans dat er iets misloopt is ook een stuk groter dan bij vorige instellingen.