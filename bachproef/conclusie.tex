%%=============================================================================
%% Conclusie
%%=============================================================================

\chapter{Conclusie}
\label{ch:conclusie}

% TODO: Trek een duidelijke conclusie, in de vorm van een antwoord op de
% onderzoeksvra(a)g(en). Wat was jouw bijdrage aan het onderzoeksdomein en
% hoe biedt dit meerwaarde aan het vakgebied/doelgroep? 
% Reflecteer kritisch over het resultaat. In Engelse teksten wordt deze sectie
% ``Discussion'' genoemd. Had je deze uitkomst verwacht? Zijn er zaken die nog
% niet duidelijk zijn?
% Heeft het onderzoek geleid tot nieuwe vragen die uitnodigen tot verder 
%onderzoek?

In het algemeen werd er niet veel nieuwe informatie ontdekt binnen het onderdeel van het experiment waarin de stappenplannen werden beschreven per testgeval. Het meeste van deze informatie is namelijk ook te vinden op het internet. De uitgeschreven stappen bieden echter wel genoeg informatie voor gebruikers die door dit onderzoek overtuigd zijn om de macht van Google in te perken op hun apparaat. Wanneer deze stappenplannen worden samengelegd met de resultaten van de monitoring van netwerkactiviteit, kan wel worden geconcludeerd of de methodes die gebruikt werden effectief invloed hadden op de hoeveelheid gecommuniceerde data.

\subsection{Testgeval 1: Android met standaard instellingen}
Uit uitgeschreven stappenplan van testgeval 1 kan afgeleid worden dat het niet veel moeite vereist om de toestand van dit geval te bekomen. Dit is namelijk ook de standaarinstelling van het apparaat. Alle instellingen kunnen op het apparaat zelf worden uitgevoerd en er is geen behoefte aan externe apparatuur of software. 

Uit de resultaten van de monitoring van de netwerkactiviteit blijkt dan ook dat een Android apparaat in deze toestand met een reeks verschillende domeinen communiceert. Een gemiddelde van 20 verzoeken per uur, zonder dat er enige interactie gebruikt, is veel. Deze verzoeken werden verstuurd naar 16 verschillende domeinen van Google. Wanneer het aantal verzoeken per uitvoering van het experiment met elkaar vergeleken worden, blijkt wel dat deze zeer veel uit elkaar liggen. Verzoeken naar domeinen zoals '\url{https://play.googleapis.com}' en '\url{https://app-measurement.com}' vertonen een gelijke trend, terwijl verzoeken naar andere domeinen soms voorkomen en soms niet voorkomen, op onregelmatige basis. Mogelijk zou dit experiment duidelijkere resultaten produceren als deze voor een langere tijd dan één uur wordt uitgevoerd.

\subsection{Testgeval 2: Android met aangepaste instellingen}
Om testgeval 2 te bereiken, moesten er bovenop het stappenplan van testgeval 1 enkele instellingen worden aangepast. Deze instellingen zelf zijn niet direct makkelijk te vinden en staan verspreid op het apparaat. Met het uitgeschreven stappenplan is het echter wel duidelijk wat er moet gebeuren. Alle instellingen kunnen op het apparaat zelf gebeuren en er is geen behoefte aan externe apparatuur of software.

De effectiviteit van deze instelling wordt duidelijk in de resultaten van de monitoring van de netwerkactiveit. Gemiddeld daalt het aantal verzoeken met 85\%. Opmerkelijk blijven verzoeken naar '\url{https://play.googleapis.com}' evenveel voorkomen als in testgeval 1. Mogelijk zijn deze afkomstig van de 'Google Play Services' applicatie, die als enigste niet uitgeschakeld kon worden. Een groot deel van de domeinen waarnaar werd gecomminiceerd in testgeval 1, komen niét meer voor in de resultaten van dit testgeval. In totaal werden er naar 5 domeinen verzoeken gestuurd, tegenover 16 domeinen in testgeval 1.

\subsection{Testgeval 3: Aangepaste versie van Android}
Binnen dit geval is het uitgeschreven stappenplan al direct groter dan deze van de andere testgevallen. Er worden binnen dit testgeval ook gebruik gemaakt van extra software, beide op het apparaat zelf en op de computer die gebruikt wordt om de software op het Android apparaat op te laden. Deze software wordt niet beschikbaar gesteld door Google of Android zelf, maar door externe ontwikkelaars. Doordat de software niet op een centrale plek wordt vrijgegeven, is het goed mogelijk dat een gebruiker een foute of illegitieme versie binnenhaalt. Foute software kan mogelijk schade veroorzaken aan de computer die gebruikt wordt, of zorgen dat het Android apparaat vastloopt en niet meer te gebruiken is. Als de juiste software gebruikt wordt, moet de gebruiker nog steeds specifieke commando's invoeren in een 'terminal' of 'cmd' venster. Voor normale gebruikers is dit niet gebruiksvriendelijk. De kans dat er iets misloopt is ook een stuk groter dan bij vorige instellingen.

Aangezien binnen dit testgeval de aanwezigheid van Google software op het apparaat volledig wordt geëlimineerd, is het ook niet verbazend dat  verzoeken naar Google domeinen niet meer voorkomen. De methodes binnen dit testgeval blijken dan ook de meest effectieve te zijn doorheen alle testgevallen. 

\section{Toekomstperspectief}
Binnen dit onderzoek werd concreet bekeken hoeveel keer een Android apparaat communiceert naar Google. Door technische beperkingen, die besproken werden in \ref{sec:metingsoftware}, was het niet mogelijk om de inhoud en specifieke URL van deze aanvragen te bekijken binnen dit onderzoek. Deze informatie zou wel zeer handig zijn voor verder onderzoek, wanneer er bijvoorbeeld wordt onderzocht welke data er precies naar Google wordt gecommuniceerd in de besproken testgevallen. In subsectie \ref{subsec:decodessl} wordt er verder ingegaan over hoe het mogelijk is om dit te doen, maar dit viel jammergenoeg buiten het bereik van dit onderzoek.