%%=============================================================================
%% Inleiding
%%=============================================================================

\chapter{Inleiding}
\label{ch:inleiding}

%TODO inleiding inleiding

%De inleiding moet de lezer net genoeg informatie verschaffen om het onderwerp te begrijpen en in te zien waarom de onderzoeksvraag de moeite waard is om te onderzoeken. In de inleiding ga je literatuurverwijzingen beperken, zodat de tekst vlot leesbaar blijft. Je kan de inleiding verder onderverdelen in secties als dit de tekst verduidelijkt. Zaken die aan bod kunnen komen in de inleiding~\autocite{Pollefliet2011}:

%\begin{itemize}
%  \item context, achtergrond
%  \item afbakenen van het onderwerp
%  \item verantwoording van het onderwerp, methodologie
%  \item probleemstelling
%  \item onderzoeksdoelstelling
%  \item onderzoeksvraag
%  \item \ldots
%\end{itemize}

\section{Probleemstelling}
\label{sec:probleemstelling}

Er is reeds veel sprake over hoeveel data Google daadwerkelijk bijhoudt over zijn gebruikers en waarom dit voor de gebruiker zelf ook negatief kan zijn. Deze data wordt verzameld wanneer men gebruik maakt van Google services die standaard meegeleverd zijn in een Android telefoon. Deze Google services zijn onder meer applicaties zoals Google Play Store, Google Maps, YouTube, etc. Ook onderliggende frameworks aangeboden door Google die andere applicaties kunnen gebruiken horen hierbij. 

Het probleem ligt hier bij het feit dat de gebruikers van deze services niet voldoende weten waarom deze data wordt bijgehouden, wat er mee gedaan wordt en waarom ze dit misschien beter zouden voorkomen. Denk in dit verband ook aan de GDPR-wetgeving die eveneens werd ingevoerd in functie van het beter informeren van de Europese burger over welke data er over hem/haar wordt opgeslagen \footnote{\url{https://gdpr-eu.be/wat-is-gdpr/}}. Google werd begin 2019 ook reeds bestraft in Frankrijk voor het overtreden hiervan, zegt het artikel (AFP, g.d.). Het artikel (Verhagen, 2019) vermeldt dat techgiganten hun dominante machtspositie misbruiken om zo zeer grote hoeveelheden informatie over hun gebruikers te weten te komen. Op deze manier kunnen ze surrealistische hoeveelheden geld verdienen met gegevens van consumenten \autocite{propken_google-data}. 

Dit werk richt zich dan ook naar Android gebruikers die belang hechten aan hun privacy, en oogt erop hun beter te informeren.

%Uit je probleemstelling moet duidelijk zijn dat je onderzoek een meerwaarde heeft voor een concrete doelgroep. De doelgroep moet goed gedefinieerd en afgelijnd zijn. Doelgroepen als ``bedrijven,'' ``KMO's,'' systeembeheerders, enz.~zijn nog te vaag. Als je een lijstje kan maken van de personen/organisaties die een meerwaarde zullen vinden in deze bachelorproef (dit is eigenlijk je steekproefkader), dan is dat een indicatie dat de doelgroep goed gedefinieerd is. Dit kan een enkel bedrijf zijn of zelfs één persoon (je co-promotor/opdrachtgever).

\section{Onderzoeksvraag}
\label{sec:onderzoeksvraag}

Dit werk zal bespreken of het kan vermeden worden dat de gebruiksdata van Android gebruikers wordt verzameld, en hoe dit kan gerealiseerd worden door Google zoveel mogelijk, al dan niet volledig, te verbannen uit hun smartphone. Daarnaast zullen ook de voordelen en nadelen die hiervan het gevolg zouden zijn, onderzocht worden.
%Wees zo concreet mogelijk bij het formuleren van je onderzoeksvraag. Een onderzoeksvraag is trouwens iets waar nog niemand op dit moment een antwoord heeft (voor zover je kan nagaan). Het opzoeken van bestaande informatie (bv. ``welke tools bestaan er voor deze toepassing?'') is dus geen onderzoeksvraag. Je kan de onderzoeksvraag verder specifiëren in deelvragen. Bv.~als je onderzoek gaat over performantiemetingen, dan 

\section{Onderzoeksdoelstelling}
\label{sec:onderzoeksdoelstelling}

Het doel van dit onderzoek is om een beeld te geven van de mogelijkheden waartoe Android gebruikers beschikken, om Google zo veel mogelijk te vermijden, en wat de beste alternatieven zijn.
%Wat is het beoogde resultaat van je bachelorproef? Wat zijn de criteria voor succes? Beschrijf die zo concreet mogelijk.

\section{Opzet van deze bachelorproef}
\label{sec:opzet-bachelorproef}

% Het is gebruikelijk aan het einde van de inleiding een overzicht te
% geven van de opbouw van de rest van de tekst. Deze sectie bevat al een aanzet
% die je kan aanvullen/aanpassen in functie van je eigen tekst.

De rest van deze bachelorproef is als volgt opgebouwd:

In Hoofdstuk~\ref{ch:stand-van-zaken} wordt een overzicht gegeven van de stand van zaken binnen het onderzoeksdomein, op basis van een literatuurstudie.

In Hoofdstuk~\ref{ch:methodologie} wordt de methodologie toegelicht en worden de gebruikte onderzoekstechnieken besproken om een antwoord te kunnen formuleren op de onderzoeksvragen.

In Hoofdstuk~\ref{ch:resultaten} zullen de resultaten worden besproken die volgden uit de experimenten van hoofdstuk \ref{ch:methodologie}.

% TODO: Vul hier aan voor je eigen hoofstukken, één of twee zinnen per hoofdstuk

In Hoofdstuk~\ref{ch:conclusie}, tenslotte, wordt de conclusie gegeven en een antwoord geformuleerd op de onderzoeksvragen. Daarbij wordt ook een aanzet gegeven voor toekomstig onderzoek binnen dit domein.

