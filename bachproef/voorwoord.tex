%%=============================================================================
%% Voorwoord
%%=============================================================================

\chapter*{\IfLanguageName{dutch}{Woord vooraf}{Preface}}
\label{ch:voorwoord}

%% TODO:
%% Het voorwoord is het enige deel van de bachelorproef waar je vanuit je
%% eigen standpunt (``ik-vorm'') mag schrijven. Je kan hier bv. motiveren
%% waarom jij het onderwerp wil bespreken.
%% Vergeet ook niet te bedanken wie je geholpen/gesteund/... heeft

Ik zit vast in het Google ecosysteem. Dagelijks gebruik ik op mijn Android telefoon applicaties die door Google ontwikkeld zijn, en zo geef ik vrijwillig mijn data weg, die zij dan kunnen gebruiken om mij gerichte advertenties te sturen en God weet wat ze er nog allemaal mee doen. De applicaties die ze aanbieden zijn zo handig en makkelijk in gebruik, maar we stellen er ons geen vragen bij hoe Google deze services gratis kan aanbieden. 

Door de jaren heen kwam kwam ik steeds meer te weten over alle data die over mij werd verzameld, maar des te meer ik er van wist, des te meer dat ik er vrede mee nam. Ik zag dit onderwerp staan op het ideeënforum van HoGent, en wist direct dat dit een intressant onderwerp zou zijn, ook al is niet iets waar ik me persoonlijk toe in staat acht. Ik ben reeds te verwikkeld in hun praktijken, maar voor anderen is dit misschien nog niet te laat. Met dit onderzoek hoop ik mensen te informeren over de hoeveeld data die precies naar Google toe wordt gecommuniceerd, en wat de opties zijn om dit tegen te gaan.

Voor ik dit hoofdstuk afrond zou ik graag nog enkele mensen bedanken. Eerst en vooral zijn dit mijn ouders, die men te alle tijde hebben gesteund en door deze opleiding hebben gesleurd. Doorheen mijn studies probeerden ze me elke examenperiode tevergeefs van mijn uitstelgedrag af te helpen, wat ik heel erg appreciëer. Ten tweede wil ik mijn co-promotor, Stein Desmet, en mijn promotor, Liesbeth Lewyllie, bedanken. Ook zij wezen me tevergeefs op mijn uitstelgedrag, maar nog belangerijker is dat ze me tijdens het schrijven van dit werk continue constructieve feedback bezorgden. Ten slotte wil ik mijn trouwe Oneplus 5 bedanken, hoewel deze geen persoon is, voor mijn misbruik te toloereren en zich door mijn experimenten te sleuren zonder (al te veel) vast te lopen.

Dit gezegd zijnde, hoop ik dat jullie dit werk intressant vinden en dat deze informatie jullie helpt om een mening te vormen over de praktijken van Google.

