%%=============================================================================
%% Voorwoord
%%=============================================================================

\chapter*{\IfLanguageName{dutch}{Woord vooraf}{Preface}}
\label{ch:voorwoord}

%% TODO:
%% Het voorwoord is het enige deel van de bachelorproef waar je vanuit je
%% eigen standpunt (``ik-vorm'') mag schrijven. Je kan hier bv. motiveren
%% waarom jij het onderwerp wil bespreken.
%% Vergeet ook niet te bedanken wie je geholpen/gesteund/... heeft

Ik zit vast in het Google ecosysteem. Dagelijks gebruik ik op mijn Android telefoon applicaties die door Google ontwikkeld zijn en zo geef ik vrijwillig mijn data weg, die Google dan kan gebruiken om mij gerichte advertenties te sturen en God weet wat ze er nog allemaal mee doen. De applicaties die Google aanbiedt zijn zo handig en makkelijk in gebruik, maar we stellen er ons geen vragen bij hoe Google deze services gratis kan aanbieden. 

Door de jaren heen kwam ik steeds meer te weten over alle data die over mij werd verzameld, maar des te meer ik er van wist, des te meer ik er vrede mee nam. Android zonder Google? Ik zag dit onderwerp staan op het ideeënforum van HoGent en wist direct dat dit een interessante bachelorproef zou opleveren, ook al is ontgooglen niet iets waar ik me persoonlijk toe in staat acht. Niet omdat ik dit technisch gezien niet zou kunnen, maar eerder omdat ikzelf reeds te veel verwikkeld ben in de praktijken van Google. Voor anderen is het echter mogelijk nog niet te laat. Met dit onderzoek hoop ik dan ook mensen te informeren over de hoeveelheid data die precies naar Google toe wordt gecommuniceerd en wat de opties zijn om dit tegen te gaan.

Voor ik dit hoofdstuk afrond, zou ik graag nog enkele mensen bedanken. Eerst en vooral zijn dit mijn ouders, die me te allen tijde hebben gesteund. Ten tweede wil ik mijn co-promotor, Stein Desmet, en mijn promotor, Liesbeth Lewyllie, bedanken. Zij bezorgden me tijdens het schrijven van dit werk continu constructieve feedback. Ten slotte wil ik mijn trouwe Oneplus 5 bedanken, hoewel deze geen persoon is, om mijn misbruik te tolereren en zich door mijn experimenten te sleuren zonder (al te veel) vast te lopen.

Dit gezegd zijnde, hoop ik dat jullie dit werk interessant vinden en dat deze informatie jullie helpt om een eigen mening te vormen over de praktijken van Google.

