%%=============================================================================
%% Samenvatting
%%=============================================================================

% TODO: De "abstract" of samenvatting is een kernachtige (~ 1 blz. voor een
% thesis) synthese van het document.
%
% Deze aspecten moeten zeker aan bod komen:
% - Context: waarom is dit werk belangrijk?
% - Nood: waarom moest dit onderzocht worden?
% - Taak: wat heb je precies gedaan?
% - Object: wat staat in dit document geschreven?
% - Resultaat: wat was het resultaat?
% - Conclusie: wat is/zijn de belangrijkste conclusie(s)?
% - Perspectief: blijven er nog vragen open die in de toekomst nog kunnen
%    onderzocht worden? Wat is een mogelijk vervolg voor jouw onderzoek?
%
% LET OP! Een samenvatting is GEEN voorwoord!

%%---------- Nederlandse samenvatting -----------------------------------------
%
% TODO: Als je je bachelorproef in het Engels schrijft, moet je eerst een
% Nederlandse samenvatting invoegen. Haal daarvoor onderstaande code uit
% commentaar.
% Wie zijn bachelorproef in het Nederlands schrijft, kan dit negeren, de inhoud
% wordt niet in het document ingevoegd.

\IfLanguageName{english}{%
\selectlanguage{dutch}
\chapter*{Samenvatting}
\lipsum[1-4]
\selectlanguage{english}
}{}

%%---------- Samenvatting -----------------------------------------------------
% De samenvatting in de hoofdtaal van het document

\chapter*{\IfLanguageName{dutch}{Samenvatting}{Abstract}}

Android telt de dag van vandaag reeds meer dan 2,5 miljard maandelijks actieve gebruikers. Android zelf is een zeer weide term. Deze bevat niet enkel het Android Open Source Project (AOSP), die de basisversie van het besturingssysteem omvat, maar ook alle aftakkingen hiervan. Zo'n aftakking wordt een 'Android skin' genoemd, omdat deze vaak ook een eigen draai geeft aan het visuele aspect van het besturingssysteem. Vaak worden er dan ook extra functionaliteiten toegevoegd waarmee smartphone producenten zich proberen te onderscheiden van elkaar. Een ander softwarepaket dat vaak wordt meegeleverd met apparaten die bestemd zijn voor de westerse markt, maar niet inbegrepen is in het AOSP, is het 'Google Apps' pakket.

Google Apps, is niet, zoals de naam impliceert, enkel een verwijzing naar Google applcaties zoals Gmail, YouTube, Maps, etc. Hieronder valt ook het onderliggende 'Google Play Services' framework. Deze biedt API's aan die makkelijker te gebruiken zijn dan degene die in het AOSP te vinden zijn, en ook meer functionaliteiten aanbieden. Doordat dit softwarepaket van Google apart staat van het besturingssysteem, kan deze geüpdatet worden zonder dat het volledige besturingssysteem moet bijgewerkt worden. Zo kunnen applicatie-ontwikkelaars zeer makkelijk verderwerken met de nieuwste functionaliteiten die de Google Play Services aanbiedt, zelf op apparaten die niet de laatste versie van Android bevatten.

Wanneer applicaties uit dit softwarepakket worden gebruikt, wordt er ook data verzameld over de gebruiker. Deze data gebruikt Google dan om gericht advertenties te kunnen sturen naar gebruikers. Volgens een onderzoek door \cite{schmidt_google-data-collection} is Android een essentiële factor bij Google's verzamelen van data. De producten die ze aanbieden zijn in staat om gebruikersgegevens te verzamelen via een verscheidenheid aan technieken die door de doorsnee gebruiker mogelijk moeilijk te begrijpen zijn. Een groot deel van de verzamelde data wordt verzamelt wanneer de gebruiker niet rechstreeks in contact komt met Google's producten. Zelf data die wordt verzameld zonder dat er een gebruiker is geïdentificeerd, kan later nog gelinkt worden aan de Google account van een gebruiker.

Applicaties binnen het Google Apps softwarepakket, worden wanneer deze worden meegeleverd met het besturingsysteem, geïnstalleerd als systeemapplicaties. Dit betekent dat ze nooit volledig verwijderd kunnen worden van het apparaat. Dit wordt een probleem wanneer de gebruiker wilt beperken welke Google software er zich op zijn/haar apparaat bevindt. Google zelf biedt reeds enkele opties aan om het verzamelen van data in te perken, maar deze zijn beperkt.

In dit onderzoek werd onderzocht wat de mogelijke manieren zijn om Google zoveel mogelijk te verbannen van het apparaat, beide door methoden die Google zelf aanbiedt en methoden die niet rechstreeks door Google ondersteund worden. Hieruit kwamen drie testgevallen voort: Android met fabrieksinstellingen, Android met aangepaste instellingen en een volledig aangepaste versie van Android. Bij elk testgeval werd bekeken welke stappen moesten worden gevolgd om de gewenste toestand van het testgeval te bereiken. Hierna werd er per geval geanalyseerd hoeveel keer er naar Google gecommuniceerd werd. 

%TODO RESULTATEN BESPREKEN
