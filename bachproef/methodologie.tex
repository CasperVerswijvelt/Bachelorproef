%%=============================================================================
%% Methodologie
%%=============================================================================

\chapter{Methodologie}
\label{ch:methodologie}

%% TODO: Hoe ben je te werk gegaan? Verdeel je onderzoek in grote fasen, en
%% licht in elke fase toe welke stappen je gevolgd hebt. Verantwoord waarom je
%% op deze manier te werk gegaan bent. Je moet kunnen aantonen dat je de best
%% mogelijke manier toegepast hebt om een antwoord te vinden op de
%% onderzoeksvraag.

%%\lipsum[21-25]
\section{Algemene opzet}
\label{sec:opzetexperiment}

Binnen dit experiment zullen er concreet 3 specifieke toestanden van een Android apparaat met elkaar vergeleken worden. Het eerste geval is een Android apparaat waarop er nog geen enkele wijziging is doorgevoerd. Deze staat dus nog gelijk met de fabrieksinstellingen van het apparaat. Bij het tweede geval zijn de instellingen zo aangepast om de dataverzameling van Google te minimaliseren. Bij deze toestand wordt er optimaal gebruik gemaakt van de mogelijkheden die Google en het Android besturingssysteem zodat er zo min mogelijk data wordt verzameld. In het derde geval zullen er methodes gebruikt worden, die niet direct door Google of Android zelf worden ondersteund, om zo Google volledig te proberen elimineren van het apparaat.

Bij deze drie gevallen zal er in detail beschreven worden welke stappen precies moeten worden genomen om de testtoestand van het apparaat te bekomen. Daarna zal het internetverkeer van elk geval geanalyseerd worden gedurende 1 uur. Concreet wordt er hier geregistreerd hoeveel keer er gemiddeld data wordt verstuurd naar specifieke internet domeinen van Google.

\section{Testapparaat}
\label{sec:testapparaat}

\section{Testgevallen}
\label{sec:testgevallen}

\section{Meting software}
\label{sec:metingsoftware}
Het verzamelen van data die de netwerkactiviteit van het apparaat beschrijft, zal op een gelijkaardige manier worden gedaan zoals in het onderzoek door \cite{schmidt_google-data-collection}, wat ook in de litatuurstudie werd besproken. Om exact te kunnen zien welke data er precies binnenkomt en buitengaat via het internet op het apparaat, is er extra software nodig. Deze optie wordt namelijk niet gegeven door het Android besturingssysteem zelf. De software die wij gebruiken moet voldoen aan enkele voorwaarden:
\begin{itemize}
    \item De verkregen data moet kunnen worden geëxporteerd naar een bruikbaar formaat.
    \item De software mag geen aanpassing maken aan of vereisen van het apparaat.
    \item De software moet de gevraagde data verzamelen zonder dat deze wordt beïnvloed door de software zelf.
\end{itemize}
Er bestaan reeds veel applicaties die rechstreeks vanaf het Android apparaat deze data kan verzamelen. Velen hiervan vallen al direct weg, aangezien ze root toegang vereisen. De 'geroote' toestand is namelijk een voorwaarde van de testgevallen, waardoor we deze doorheen de gevallen niet ingeschakeld kunnen houden.
