%---------- Inleiding ---------------------------------------------------------

\section{Introductie} % The \section*{} command stops section numbering
\label{sec:introductie}

%Hier introduceer je werk. Je hoeft hier nog niet te technisch te gaan.

%Je beschrijft zeker:

%\begin{itemize}
%  \item de probleemstelling en context
%  \item de motivatie en relevantie voor het onderzoek
%  \item de doelstelling en onderzoeksvraag/-vragen
%\end{itemize}
Het is reeds zeer actueel dat er heel veel persoonlijke data wordt bijgehouden bij het gebruiken van online services. Het probleem ligt dan ook bij het feit dat de gebruikers van deze services niet voldoende weten waarom deze data wordt bijgehouden, wat er mee gedaan wordt en waarom ze dit misschien beter zouden voorkomen. Denk in dit verband ook aan de GDPR wetgeving die eveneens werd ingevoerd in functie van het beter informeren van de Europese burger over welke data er over hem/haar wordt opgeslagen. 

\vspace{2mm}

In mijn werk zal ik dieper ingaan op wat er precies wordt bijgehouden van data bij het gebruik van het Android besturingssysteem, waarom dit gebeurt, en hoe het vermeden kan worden. Dit kadert in de bescherming van de persoonlijke levenssfeer. 

\vspace{2mm}

\noindent De onderzoeksvragen voor mijn werk zijn dan als volgt:


\begin{itemize}
  \item Welke data wordt er bij het gebruik van het Android besturingssysteem exact bijgehouden door Google?
  \item Welke opties worden reeds door Google aangeboden om het delen van deze data te beperken?
  \item Kan Google volledig verbannen worden uit het Android besturingssysteem, en hoe?
  \item Zijn er voldoende alternatieven om de functionaliteiten die Google biedt te vervangen, en welke zijn deze?
  \item Zijn Android gebruikers voldoende ingelicht over de hoeveelheid data dat ze delen met Google?
\end{itemize}


%---------- Stand van zaken ---------------------------------------------------

\section{Stand van zaken}
\label{sec:state-of-the-art}

%Hier beschrijf je de \emph{state-of-the-art} rondom je gekozen onderzoeksdomein. Dit kan bijvoorbeeld een literatuurstudie zijn. Je mag de titel van deze sectie ook aanpassen (literatuurstudie, stand van zaken, enz.). Zijn er al gelijkaardige onderzoeken gevoerd? Wat concluderen ze? Wat is het verschil met jouw onderzoek? Wat is de relevantie met jouw onderzoek?
Aangezien Android eigendom is van Google, is het ook vanzelfsprekend dat Google heeft gezorgd dat de Google services zeer nauw geïntegreerd zijn in, zelfs verweven, in het smartphone besturingssysteem Android. Deze nauwe integratie heeft hierdoor dan ook als nadeel dat Google toegang heeft tot alle informatie over wat je via deze services doet.
%TODO: Vast in google ecosysteem

\vspace{2mm}

Over dit onderwerp werden reeds verschillende artikels geschreven vanuit diverse invalshoeken, evenwel zonder  een totaalbeeld te vormen. Mijn doelstelling is om deze problematiek zelf systematisch en gestructureerd te onderzoeken.

%TODO: evenwel / maar ?
%Verwijs bij elke introductie van een term of bewering over het domein naar de vakliteratuur, bijvoorbeeld~\autocite{Doll1954}! Denk zeker goed na welke werken je refereert en waarom.

% Voor literatuurverwijzingen zijn er twee belangrijke commando's:
% \autocite{KEY} => (Auteur, jaartal) Gebruik dit als de naam van de auteur
%   geen onderdeel is van de zin.
% \textcite{KEY} => Auteur (jaartal)  Gebruik dit als de auteursnaam wel een
%   functie heeft in de zin (bv. ``Uit onderzoek door Doll & Hill (1954) bleek
%   ...'')

%Je mag gerust gebruik maken van subsecties in dit onderdeel.

%---------- Methodologie ------------------------------------------------------
\section{Methodologie}
\label{sec:methodologie}

%Hier beschrijf je hoe je van plan bent het onderzoek te voeren. Welke onderzoekstechniek ga je toepassen om elk van je onderzoeksvragen te beantwoorden? Gebruik je hiervoor experimenten, vragenlijsten, simulaties? Je beschrijft ook al welke tools je denkt hiervoor te gebruiken of te ontwikkelen.
Eerst en vooral zal ik bekijken welke data er allemaal wordt bijgehouden bij het gebruik van een Android telefoon.

\vspace{2mm}

Vervolgens zal ik nagaan welke opties om het delen van data te beperken ons worden aangeboden vanuit het besturingssysteem zelf. Hier gaat het nog niet over het 'verbannen' van Google. Of dit beperkingen zou leggen op bepaalde functionaliteiten van Android wordt natuurlijk ook onderzocht.

\vspace{5mm}

Daarna zal ik bestuderen of het mogelijk is Google volledig te verbannen uit de wortels van het Android besturingssysteem, door middel van methodes die niet direct door Android en Google zelf worden ondersteund.

\vspace{2mm}

Wanneer je niet meer met Google services werkt, betekent dit uiteraard dat je veel functionaliteit zal mislopen, aangezien dit niet het beoogde gebruik is volgens Google. Ik zal onderzoeken hoe deze verloren functionaliteiten zoveel mogelijk terug ingevuld kunnen worden door alternatieven, en in welke mate deze alternatieven opwegen tegen de standaard implementatie.

\vspace{2mm}

Aangezien reeds veel van deze informatie beschikbaar is op het internet, zullen voorgenoemde onderzoeken voornamelijk gebeuren via opzoekwerk en verificatie of deze informatie ook klopt, door middel van eigen experimenten.

\vspace{2mm}

Naast deze onderzoeken, overweeg ik ook een enquête uit te voeren waarbij ik mensen vraag of ze op de hoogte zijn welke data ze reeds met Google delen door het gebruiken van hun Android telefoon.

%---------- Verwachte resultaten ----------------------------------------------
\section{Verwachte resultaten}
\label{sec:verwachte_resultaten}

%Hier beschrijf je welke resultaten je verwacht. Als je metingen en simulaties uitvoert, kan je hier al mock-ups maken van de grafieken samen met de verwachte conclusies. Benoem zeker al je assen en de stukken van de grafiek die je gaat gebruiken. Dit zorgt ervoor dat je concreet weet hoe je je data gaat moeten structureren.
Ik verwacht dat Google een zeer grote hoeveelheid gebruiksdata zal bijhouden per gebruiker. Dit zou dan gaan over alles wat je doet via google services, zoals locatiegegevens, items die je hebt opgezocht, welke apps je allemaal staan hebt, etc.

\vspace{2mm}

Bij de vraag of Google zelf al mogelijkheden biedt om het delen van data te beperken denk ik dat Google in Android zelf al heel wat mogelijkheden geeft om dit te doen. Ik geloof echter niet dat deze opties de gebruiker genoeg controle zullen geven. Google zal hiernaast nog steeds over een zeer grote hoeveelheid data van deze gebruiker blijven beschikken.

\vspace{2mm}

Aangezien de Android development community zeer groot is, denk ik dat het zeer goed mogelijk is dat er reeds manieren ontwikkeld zijn om Google volledig te verbannen uit Android. Als dit mogelijk is, zullen er volgens mij zeker ook methodes bestaan om alle functionaliteiten die we verliezen door dit te doen, te vervangen door alternatieven. Ik verwacht niet dat deze alternatieven even gestroomlijnd zullen zijn als het aanbod van Google. 

\vspace{2mm}

Wat betreft mijn laatste onderzoeksvraag, denk ik niet dat mensen voldoende ingelicht zijn over de hoeveelheid data die ze met Google delen. Ik ga hiervan uit aangezien ik mezelf bestempel als een Android 'power user', en ikzelf eerder ook niet wist wat er juist werd bijgehouden.


%---------- Verwachte conclusies ----------------------------------------------
\section{Verwachte conclusies}
\label{sec:verwachte_conclusies}

%Hier beschrijf je wat je verwacht uit je onderzoek, met de motivatie waarom. Het is \textbf{niet} erg indien uit je onderzoek andere resultaten en conclusies vloeien dan dat je hier beschrijft: het is dan juist interessant om te onderzoeken waarom jouw hypothesen niet overeenkomen met de resultaten.

Uit mijn verwachte resultaten zou ik concluderen dat Google een heel stuk meer gebruiksinformatie ter beschikking krijgt via haar smartphone besturingssysteem Android. De gebruikers van dit besturingssysteem zijn hiervan onvoldoende op de hoogte en zien bijgevolg hiervan ook de ernst niet in.

Een goed begin om de macht van Google op dit vlak wat in te perken zullen de opties zijn die Android ons zelf al aanbiedt. Dit zullen waarschijnlijk niet veel opties zijn, maar alles telt natuurlijk. 

Om nog verder te gaan zullen er manieren zijn om het besturingssysteem te modifiëren zodat Google volledig verbannen wordt uit het besturingssysteem. Deze manieren zijn grotendeels afkomstig van Android gebruikers van de development community, die eveneens niet akkoord zijn met de hoeveelheid data die Android ophaalt en doorstuurt.

%TODO : Spellcheck!!!

%TODO: Dependency op google services, ecosysteem?
